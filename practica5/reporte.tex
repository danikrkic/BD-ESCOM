% !TeX spellcheck = es_ES
\documentclass[12pt, titlepage]{article}
\usepackage[utf8]{inputenc}
\usepackage[spanish]{babel}
\usepackage{float}
\usepackage[letterpaper, margin=2.5cm]{geometry}
\usepackage[nottoc,notlot,notlof]{tocbibind} % Hace que se agregen las referencias al indice
\usepackage{url}
\usepackage{graphicx} 
\usepackage{listings}
\usepackage{color}
\definecolor{dkgreen}{rgb}{0,0.6,0}
\definecolor{gray}{rgb}{0.5,0.5,0.5}
\definecolor{mauve}{RGB}{253,151,31}

\lstset{frame=tb,
    language=Sql,
    aboveskip=3mm,
    belowskip=3mm,
    showstringspaces=false,
    columns=flexible,
    basicstyle={\small\ttfamily},
    numbers=none,
    numberstyle=\tiny\color{gray},
    keywordstyle=\color{blue},
    commentstyle=\color{dkgreen},
    stringstyle=\color{mauve},
    breaklines=true,
    breakatwhitespace=true,
    tabsize=2,
    morekeywords={use}
}

\title{Reporte: Práctica 5}
\author{Carlos Tonatihu Barrera Pérez \\ Profesor: Hernández Contreras Euler \\ Bases de Datos \\ Grupo: 2CM1 }
\date{31 de marzo de 2017}

\begin{document}
	\maketitle
	\tableofcontents
	\section{Marco Teórico}
	\section{Desarrollo}
	Como es costumbre se importo el contenido de la base de datos con la que se iba a trabajar desde un archivo .sql y se procedió a realizar las siguientes consultas.
	
	Primero se mostró el número de TT de aquellos que fueron dirigidos por Andres Ortigoza.
	
	Después se despego toda la información de aquellos TTs que tienen en su titulo "redes neuronales".
	
	Lo siguiente fue mostrar dictamen y la calificación de los sinodales de aquellos tts que ha dirigido la profesora Fabiola Ocampo.
	
	A continuación se mostró el numero de TT de aquellos profesores que se apellidan Martínez y se incluyo el nombre completo del profesor.
	
	Después, se imprimió el grado de estudios que tienen los profesores que se apellidan Maldonado.
	
	La siguiente acción fue mostrar cuales son los TTs que han reprobado ademas de desplegó sus respectivos directores.
	
	Luego se desplegó que TTs se han presentado en el año 2009 junto con sus directores.
	
	A continuación se imprimió que sinodales tiene el TT 2010-0046.
	
	Lo siguiente fue desplegar cuantos TTs ha dirigido el profesor Euler.
	
	Ademas, se imprimió el nombre de estos TTs.
	
	Y por ultimo se mostró a los sinodales de dichos TTs.
	\section{Conclusiones}
	Poco a poco las consultas se vuelven más complejas y la ultima realizada en esta práctica es una prueba de ello es por esto que se debe de continuar así para que este tipo de ejercicios no resulten ser un problema y se pueden hacer de la manera más rápida y eficiente posible ya que a la larga el que tan fácil podamos resolver un problema de este tipo determinara la complejidad que podamos alcanzar en los sistemas que se deseen implementar siempre y cuando estos necesiten una base de datos.
	\bibliography{bibliografia} 
	\bibliographystyle{ieeetr}
\end{document}